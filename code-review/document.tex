\documentclass{tufte-handout}

\usepackage{hyperref}
\usepackage{fancyhdr}
\usepackage{array}
\usepackage[export]{adjustbox}
\usepackage{setspace}
\usepackage{xcolor}

% 
\usepackage{booktabs}
% For merged cells
\usepackage{multirow}	
\usepackage{array}

\usepackage{librecaslon}


\title{Pre-Publication Code Review Checklist}
\date{}

% Set header and footer
\pagestyle{fancy}
\fancyhf{}
\lfoot{\includegraphics[height=1.3cm,keepaspectratio]{../img/i2i}}
\cfoot{\includegraphics[height=1.3cm,keepaspectratio]{../img/wb}}
\rfoot{\includegraphics[height=1.3cm,keepaspectratio]{../img/analytics}}

% Put checkbox to the left of the text
\def\LayoutCheckField#1#2{% label, field
	#2 #1%
}

% Line spacing
\onehalfspacing

% Define DIME Analytics visual identity colors
\definecolor{fontcolor}{HTML}{7A0569}

\titleformat{\section}%
{\Large\rmfamily\bf\color{fontcolor}}% format applied to label+text
{\llap{\colorbox{fontcolor}{\parbox{1.5cm}{\hfill\huge\color{fontcolor}\thesection}}}}% label
{2pt}% horizontal separation between label and title body
{}% before the title body
[]% after the title body

% ------------------------------ End of preamble ---------------------------------------------
\begin{document}
	\begin{fullwidth}
		
	\maketitle
	
	% Please add the following required packages to your document preamble:
	
	\begin{table}[h!]
		\singlespacing
		\begin{tabular}{|>{          \arraybackslash} m{3.5cm} 
				 	    |>{\centering\arraybackslash} m{2.5cm} 
			 	        |>{\centering\arraybackslash} m{3.3cm}
		 	            |>{          \arraybackslash} m{6cm}|}
		% Title
			\multicolumn{4}{c}{\textbf{\large{\textcolor{fontcolor}{TOP Guidelines Compliance at DIME}}} (https://cos.io/top/)} \\  
			\multicolumn{4}{c}{}\\ \hline
		% Header
			  \hspace{.8cm}\textbf{Category}
			& \vspace{2mm}\textbf{Required TOP Level}\vspace{1mm}
			& \vspace{2mm}\textbf{DIME Suggested Approach}\vspace{1mm}
			& \hspace{1.3cm}{\textbf{Implementation}} \\ \hline
		% ------	
			  Citation Standards	
			& \small{Level III}
			& \vspace{2mm}{\small Harvard Dataverse Style}\vspace{1mm}
			& \vspace{2mm}{\small Article must include formal citations for data and documentation.}\vspace{1mm}	\\ \hline	
		% ------	
			  \vspace{3mm}Data Transparency\vspace{3mm}
			& \vspace{3mm}\small{Level III}\vspace{3mm}
			& \vspace{.4cm}\multirow{2}{\hsize}{\small{Microdata Catalog (instructions)}}
			& \multirow{2}{\hsize}{\small Data, code, and documentation (such as survey instruments) must be posted to a trusted repository (which can be private or embargoed.)}  \\ \cline{1-2}
		% ------
			  \vspace{3mm}{Documentation}\vspace{2mm}
			& \small{Level III}
			& 
			&  \\ \cline{1-3}
		% ------
			  Code Transparency
			& \small{Level III}
			& \small{GitHub Repository}
			& \small{Reported analyses will be reproduced independently prior to publication.}\vspace{1mm} \\ \hline
		%
			  Design Transparency
			& \small{Level II}
			& \multirow{3}{\hsize}{{\small{AEA, RIDIE, or other registry, in combination with GitHub Issues or other transparent record-keeping of decisions taken during project and analysis. (AEA policy)}}}
			& \vspace{2mm}{\small{Experimental design, including sampling and randomization, must be fully described. Analytical choices (statistical model, covariates, etc.) must be documented.}}\vspace{1mm} \\ \cline{1-2}
		% ------
			  Study Preregistration
			& \small{Level II}
			& 
			& \vspace{2mm}{\small{State whether the study is preregistered, and, if so, allow access during peer review.}}\vspace{1mm} \\ \cline{1-2} 
		% ------
			  Pre-Analysis  Plan
			& \small{Level II}
			& 
			& \vspace{2mm}\small{State whether a pre-analysis plan was registered, and, if so, allow access during peer review.}\vspace{1mm} \\ \hline
		% ------
			  Replication
			& \vspace{2mm}\small{Level III {(replication studies only)\vspace{1mm}}}
			& \small{JDE Registered Reports or other}
			& \vspace{2mm}{\small Registered Reports for replication studies with peer review before observing outcomes can be submitted.}\vspace{1mm} \\ \hline
		\end{tabular}
	\end{table}
		
	\newpage
	\section*{Part 1: Computational Reproducibility}
	
		\begin{Form}
			\CheckBox[height=0.01cm, width=0.4cm]{Entire project directory provided as a .zip file or GitHub link \textbf{(required)}}
		\end{Form}
			
		
		\vspace{.3cm}
		\noindent \textbf{The directory includes:}
		
		
		\begin{Form}
			\CheckBox[height=0.01cm, width=0.4cm, bordercolor=gray]{ All necessary data for the analysis. \textbf{(required)}}
		\end{Form}
		
		\begin{Form}
			\CheckBox[height=0.01cm, width=0.4cm, bordercolor=gray]{ All code necessary for the analysis. \textbf{(required)}}
		\end{Form}
		
		\begin{Form}
			\CheckBox[height=0.01cm, width=0.4cm, bordercolor=gray]{ The raw outputs used for the paper (e.g. tables and figures). \textbf{(required)}}
		\end{Form}
		
		\begin{Form} 	
			\CheckBox[height=0.01cm, width=0.4cm, bordercolor=gray]{ Fully de-identified data. \textbf{(required without pre-agreed exception)}}
		\end{Form}
		
		\begin{Form}
			\CheckBox[height=0.01cm, width=0.4cm, bordercolor=gray]{ No documentation, code, outputs or datasets not used in the paper.}
		\end{Form}
		
		
		\vspace{.3cm}
		\noindent \textbf{Master script:}
		
		\begin{Form}
			\CheckBox[height=0.01cm, width=0.4cm]{ The master script (e.g. .do file, R script) runs the entire project when executed. \textbf{(required)}}
		\end{Form}
		
		\begin{Form}
			\CheckBox[height=0.01cm, width=0.4cm, bordercolor=gray]{ Is located in the root directory.}
		\end{Form}
		
		\begin{Form}
			\CheckBox[height=0.01cm, width=0.4cm, bordercolor=gray]{ Runs if reviewer changes only directory path(s) in one location of the master script. \textbf{(required)}}
		\end{Form}
		
		\begin{Form}
			\CheckBox[height=0.01cm, width=0.4cm, bordercolor=gray]{ Re-creates all raw outputs exactly as they appear in the paper. \textbf{(required)}}
		\end{Form}
		
		\vspace{.2cm}
		\noindent Indicate the master script filename and line number where directory path/paths should be changed: \\
		\TextField[height=0.4cm, width = 13cm]{File name:} \\ 
		\TextField[height=0.4cm, width = 13cm]{Line:\,\,\,\,\,\,\,\,\,\,\,\,\,\,\,}
		
		\vspace{.4cm}
		\noindent \textbf{Reproducibility:}
		
		\begin{Form}
			\CheckBox[height=0.01cm, width=0.4cm, bordercolor=gray]{ All code runs completely on a new computer. \textbf{(required)}}
		\end{Form}
		
		\begin{Form}
			\CheckBox[height=0.01cm, width=0.4cm, bordercolor=gray]{ Any required user-written commands are installed as part of the master script (e.g. using ssc install, net install or R code for installing packages). The appropriate version of the package is installed when versioning affects the results or the code.}
		\end{Form}
		
		\begin{Form}
			\CheckBox[height=0.01cm, width=0.4cm, bordercolor=gray]{ Critical settings like version, matsize, and varabbrev are specified. The master file either sets these directly or uses a wrapper command (e.g. ieboilstart from ietoolkit).}
		\end{Form}
		
		\vspace{.3cm}
		\noindent \textbf{Outputs (tables and graphs):}
		
		\begin{Form}
			\CheckBox[height=0.01cm, width=0.4cm, bordercolor=gray]{ All outputs are reproduced exactly as they appear in the paper. (See AEA policy) \textbf{(required)}}
		\end{Form}
		
		\begin{Form}
			\CheckBox[height=0.01cm, width=0.4cm, bordercolor=gray]{ Each output clearly corresponds by name to an exhibit in the paper, and vice versa (supplying a compiling TeX document can support this). \textbf{(required)}}
		\end{Form}
		
		\begin{Form}
			\CheckBox[height=0.01cm, width=0.4cm, bordercolor=gray]{ The outputs are produced in the same folder structure that is saved in the submission package.}
		\end{Form}
		
		\begin{Form}
			\CheckBox[height=0.01cm, width=0.4cm, bordercolor=gray]{ The submission package includes code to create all in-text numerical citations that are not drawn directly from tables and figures.}
		\end{Form}
		
		\vspace{.2cm}
		\noindent Approximately how long does the code take to run (i.e., minutes, hours, or days)? \noindent \TextField[height=0.4cm, width = 3cm]{} \\ 
	
	
	\section*{Part 2: Repository Access Information}
	
		\begin{tabular}{lc}
		\TextField[width = 9.1cm]{Microdata Catalog entry:} & \CheckBox[width=0.4cm, bordercolor=gray]{ Public} \\
		\TextField[width = 10.1cm]{GitHub repository:} 		& \CheckBox[width=0.4cm, bordercolor=gray]{ Public} \\
		\TextField[width = 10.5cm]{Pre-registration:} 		& \CheckBox[width=0.4cm, bordercolor=gray]{ Public} \\
		\TextField[width = 10.3cm]{Pre-analysis plan:} 		& \CheckBox[width=0.4cm, bordercolor=gray]{ Public} \\
		\TextField[width = 9.55cm]{Other documentation:}	& \CheckBox[width=0.4cm, bordercolor=gray]{ Public} \\
		\end{tabular}
	
	
	\section*{Part 3: Ease of Use}
	
		These practices are recommended but not required. This paper incorporates those indicated below:
		\vspace{.3cm}
		
		\noindent
		\begin{Form}
		\CheckBox[height=0.01cm, width=0.4cm, bordercolor=gray]{ Scripts for data cleaning, variable creation, and analysis are separate}
		\end{Form}
		
		\noindent
		\begin{Form}
		\CheckBox[height=0.01cm, width=0.4cm, bordercolor=gray]{ Variable creation for derived or constructed measures is done in separate scripts from data analysis and includes detailed code comments about each new variable that is generated.}
		\end{Form}
		
		\noindent
		\begin{Form}
		\CheckBox[height=0.01cm, width=0.4cm, bordercolor=gray]{ Analysis scripts do not include any data cleaning or variable creation, unless necessary for the creation of a table or graphic.}
		\end{Form}
		
		\noindent
		\begin{Form}
		\CheckBox[height=0.01cm, width=0.4cm, bordercolor=gray]{ Analysis scripts are completely modular: they do not depend on having the results of other script in memory, unless that other script is the Master script. Each script starts by loading all necessary data.}
		\end{Form}
		
		\noindent
		\begin{Form}
		\CheckBox[height=0.01cm, width=0.4cm, bordercolor=gray]{ Separate scripts are provided for each exhibit.}
		\end{Form}
		
		\noindent
		\begin{Form}
		\CheckBox[height=0.01cm, width=0.4cm, bordercolor=gray]{ All code is well-commented and formatted, such that one can easily identify functional chunks of code and evaluate whether they correctly implement the econometric or statistical process described.}
		\end{Form}
		
		\noindent
		\begin{Form}
		\CheckBox[height=0.01cm, width=0.4cm, bordercolor=gray]{ Graphics are output as .eps files or other vector images when possible.}
		\end{Form}
		
		\noindent
		\begin{Form}
		\CheckBox[height=0.01cm, width=0.4cm, bordercolor=gray]{ Tables are output as .csv or .tex files or other raw text files when possible.}
		\end{Form}
		
		\noindent
		\begin{Form}
		\CheckBox[height=0.01cm, width=0.4cm, bordercolor=gray]{ In-text numerical citations (other than those drawn directly from tables and figures) are computed and recorded in a dynamic document format like .ipynb, .stmd using markstat in Stata, or .Rmd using R.}
		\end{Form}
	
	\end{fullwidth}
\end{document}